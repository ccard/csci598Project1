\documentclass[11pt,conference]{IEEEtran}
\usepackage[utf8]{inputenc}
\usepackage[english]{babel}

\begin{document}

\section{Introduction(Chris Card)}

Using robots in healthcare is a difficult and controversial subject due to many factors, some of the more prominent being that people like having a human presence when their internal organs are being worked on.  However, robotics can greatly improve the effectiveness of healthcare while reducing the cost.  Robots can accomplish this because they can reduce cognitive loads on doctors and nurses reducing the number of 
mistakes made.  They can also improve the quality of surgery and post op care through minimally invasive surgical and therapy techniques that reduce infection rates and improve recovery time.  Robots also increase the number of patients that a doctor or nurse can treat at one time because they free them from tasks that take time away from patient care.  These are tasks that the healthcare industry is continuously trying to improve upon.
\newline
\indent For example, robotics can improve the effectiveness of invasive surgical procedures by making the procedures less invasive and more effective than tradition methods.  One robot that does this is the HeartLander robot which is placed on the heart through a small hole in 
the chest and then travels to the desired region of the heart to provide the therapy. It is also capable of moving to different areas of the heart without the need for new incisions.  The HeartLander can also perform the 
operations with out the need to deflate the lungs or stop the heart.  This improves recovery times while reducing infection rates, making it a more effective method for heart surgery then the traditional method.  
\newline
\indent Another example of how robots can improve healthcare is in general checkups and post ops in the hospital.  This would typically require a doctor to make regular time consuming rounds of a hospital to perform simple checkups.  This roll can be filled with an autonomous virtual presence robots like i-Robots RP-VITA.  This robot would allow doctors to send the robot to different patients from their tablets or computers allowing them to continue to work on others things.  Once the robot arrived a nurse could hook up instruments (\textit{i.e.} stethoscopes, ultra-sounds, pressure cuffs, \textit{etc.}).  This system would reduce the amount of time the doctor needs to make rounds in the hospital while still providing face to face communication with the patients.
\newline
\indent These examples bring up many challenges for robotics in this domain. One of the challenges that robots face is that people like meeting with a doctor or surgeon face to face because it reassures them that their needs are being heard and that the doctor cares.  This is especially true when people are worried about a procedure or their wellbeing.  Some may feel that the doctor will do a better job than the robot, even though the robot is guided by the doctor and may be more effective and safe.  This stigma is something that will need to be 
overcome for robotics to be accepted.  Another difficulty for robots is safety.  The healthcare industry has very strict safety regulations that must be complied with and failure to comply with the regulations can cost the hospital millions and/or cause loss of life.  One of the most difficult challenges to overcome is the dexterity of the robot.  Many applications in this field require dexterous action to be taken, \textit{i.e.} sewing a wound shut, setting a bone, organ transplants, \textit{etc.}.  All of these challenges will be difficult to overcome. 
\newline
\indent The rest of this paper will explore the questions: what makes a robot particularly suited to this application, how do robots perceive and react to the environment, are they intelligent and can they adapt, and what are their limitations.  These questions will be explored in two areas of the healthcare system, surgery and in hospital general checkups using robots that exist in these fields.  For the surgical field the HeartLander surgical robot will be used as an example and in the hospital general checkups i-Robot RP-VITA autonomous telepresence robot will be used. 
\end{document}
