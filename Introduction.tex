\documentclass[11pt,conference]{IEEEtran}
\usepackage[utf8]{inputenc}
\usepackage[english]{babel}

\begin{document}

\section{Introduction(Chris Card)}

Using robots in healthcare is a difficult and controversial subject due to many factors, some of the more prominent being that people like having a human presence when their internal organs are being worked on.  However, robotics can greatly improve the effectiveness of healthcare while reducing the cost.  Robots can accomplish this because they can reduce cognitive loads on doctors and nurses reducing the number of mistakes made.  They can also improve the quality of surgery and post op care by reducing infection rates and minimally invasive surgery.  Robots also increase the number of patients that a doctor or nurse can treat at one time because they free them from tasks that take time away from patient care.  These are tasks that the healthcare industry is continuously trying to improve upon.
\newline
For example, robotics can improve the effectiveness of invasive surgical procedures that would normally require a doctor to be flown in and a lager incision made so the surgeon can see and access the region he is operating on.  The introduction of laparoscopic procedures these are procedures where the surgeon creates a small incision in the skin to perform the operation through.  This method of surgery has the benefit of improving recovering time and reducing infection rates because the incisions are much smaller than traditional surgery.  Laparoscopic surgery is difficult to perform because of the size of the incision and the surgeon’s ability to see into incision.  This is where robotics can benefit surgery because robots can be equipped with fine and precisely controllable manipulators with cameras on the end of them.  This allows the surgeons to more easily operate through the small incisions.  Robots can also be operated remotely, so surgeons don’t need to be flown in from around the world.  This improves both recovery times for the patients and reduces the cost of surgery because smaller incisions are needed and surgeons don’t need to travel to perform the operation.  An example of a robot that fills this need and is currently in use is the Da Vinci robot.
\newline
Another example of how robots can improve healthcare is in general checkups and post ops in the hospital.  This would typically require a doctor to make regular time consuming rounds of a hospital to perform simple checkups.  This roll can be filled with autonomous virtual presence robots like i-Robots RP-VITA.  This robot would allow doctors to send the robot to different patients from their tablets or computers allowing them to continue to work on others things.  Once the robot arrived a nurse could hook up instruments (\textit{i.e.} stethoscopes, ultra-sounds, pressure cuffs, \textit{etc.}).  This system would reduce the amount of time the doctor needs to make rounds in the hospital while still providing face to face communication with the patients.
\newline
This does bring up a challenge for robotics in this domain is that people like meeting with a doctor or surgeon face to face because it reassures them that their needs are being heard and that the doctor cares.  This is especially true when people are nervous or worried about a procedure or their wellbeing.  Some may feel that the doctor will do a better job than the robot even though the robot is guided by the doctor and may be more effective and safe.  This stigma is something that will need to be overcome for robotics to be accepted.  Another difficulty for robots is safety.  The healthcare industry has very strict safety regulations that must be complied with and failure to comply with the regulations can cost the hospital millions and/or cause loss of life. One of the most difficult challenges to overcome is the dexterity of the robot.  Many applications in this field require dexterous action to be taken, \textit{i.e.} sewing a wound shut, setting a bone, organ transplants, \textit{etc.}.  All of these challenges will be difficult to overcome. 
\newline
The rest of this paper will explore the questions: what makes a robot particularly suited to this application, how do robots perceive and react to the environment,  are they intelligent and can they adapt, and what are their limitations.  These questions will be explored into areas of the healthcare surgery and in hospital general checkups using robots that exist in these fields.  For the surgical field the Da Vinci surgical robot will be used as an example and for the in hospital general checkups the i-Robot RP-VITA autonomous telepresence robot will be used. 
\end{document}
