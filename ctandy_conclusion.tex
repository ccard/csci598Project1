\documentclass[11pt,conference]{IEEEtran}
\usepackage[utf8]{inputenc}
\usepackage[english]{babel}

\begin{document}
\section{Conclusion(Charles Tandy)}

\indent What does the future hold for robots in the healthcare industry. We have taken a look at two prominent robots. The RP-Vita and the Heartlander. These robots have proven to be great assets in the healthcare industry. The RP-Vita has made healthcare much more efficient. While the Heartlander with it's minimally invasive surgeries will greatly decrease the number of complications that can arise from heart surgery. These robots cover two very important aspects of healthcare. Just a few years ago robots like the RP-Vita and the Heartlander would be something from science-fiction. So, what does the future hold for robots in the healthcare industry?
\newline
\indent Given current and emerging technologies we can begin to speculate about the perfect robot in the healthcare industry. We already see that the RP-Vita can move through hospitals autonomously, the next step for this robot would be the ability to diagnose patients. With the capability to diagnose patients autonomously, patients could be treated much more rapidly and with a high degree of accuracy with their diagnosis. Automation would also be extremely beneficial in surgical applications. The Heartlander has only scratched the surface of automated surgical practices. With the rapid development of technology it might not be long until we can have fully automated surgeries. The full use of automation in surgical applications would mean much fewer complications. This would also virtually remove human error from the equation. Additionally, people would be able to get into surgeries more rapidly. The results could mean the lives of millions of people being saved. 
\newline
\indent While automation could potentially save millions of lives, there may be other things to consider.

\end{document}