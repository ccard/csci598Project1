\documentclass[11pt,conference]{IEEEtran}
\usepackage[utf8]{inputenc}
\usepackage[english]{babel}

\begin{document}

\section{RP-VITA Part 2 (Ryan Langewisch)}
One primary characteristic that is often associated with robots is intelligence. The term "artificial intelligence" is prominent in the robotics field, indicating that the intelligence that we find in the realm of robotics is not true conscious intelligence, but instead an emulation of human intelligence. In the context of human centered robotics, intelligence really boils down to the robot's ability to exemplify human like decision making skills. In the case of the RP-VITA robot, independently intelligent functionality is actually rather limited. 
\newline
\indent The majority of the benefit that the RP-VITA robot provides is in the form of allowing a doctor to work remotely without actually traveling to the patient's room. This in itself does not require much artificial intelligence, but rather it simply entails having the appropriate technology for remote access and control that is found in many modern computers. The component of the RP-VITA robot's skill set that does truly exemplify the imitation of human intelligence is its movement. One of the things that makes the RP-VITA robot so useful is that it can reach the appropriate location in the hospital without any human guidance. This requires the robot to be aware of the hospital layout, and then use path-finding algorithms to determine how to reach a target location from any other location in the hospital. This task is complicated by the fact that a hospital is full of many other moving entities, most of which take priority over the RP-VITA robot. As a result, the robot needs to be capable of adjusting its path to stay out of the way, while still progressing towards its target location. It is this capability that really moves RP-VITA into the category of "robots" rather than just being a computer. If the RP-VITA robot was limited to a static location in one of the hospital's rooms, or required human interaction to be moved, it is more likely that it would just be considered a computer or another medicinal machine.
\newline
\indent Having discussed many of the features and capabilities of the RP-VITA robot, it is appropriate to also consider its limitations. The robot's primary purpose of allowing a doctor to easily have remote access to a patient also proves to be its biggest limitation: the robot is only as effective as the doctor that is available to be using it. If the doctor is busy and cannot remote in to the device, the RP-VITA robot can no longer provide the services that it is designed for. This somewhat goes against the central ideal of Human Centered Robots, in that the robot is still highly dependent on human control to be effective. That said, the purpose of the RP-VITA robot is not to automate and emulate a doctor's actions once in the patient's room, but rather it is designed to navigate to the correct location efficiently and then provide a doctor with the access and tools necessary to help him do his job effectively from a distance. With this scope in mind, the RP-VITA robot does indeed meet its goals as a human centered robot. The limitations when it comes to independent decision making and intelligence during interaction with the patient are certainly areas of improvement for future designs, but not inherently a flaw in the design of the RP-VITA robot itself.
\end{document}