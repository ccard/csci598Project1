\documentclass[11pt,conference]{IEEEtran}
\usepackage[utf8]{inputenc}
\usepackage[english]{babel}

\begin{document}

\section{ HeartLander Part 1 (Timm Nygren)}

\indent The HeartLander robot is a miniature robot that navigates along the surface of the heart delivering minimally invasive therapy \cite{overview}. Several characteristics of the HeartLander robot make it suitable for cardiac surgery. A subset of these characteristics can help define this robot as a human centered robot. The HeartLander robot is suitable for applications in healthcare due to many features from its size and unique worm-like movement to possessing the ability to sense the environment and manipulate tools for epicardial interventions. Other characteristics include the ability to move autonomously and provides an increase in safety over conventional methods used today.
\newline
\indent An important characteristic for any robot in the medical field is size. This  miniature robot is comprised of two sections, a front and rear body section, that measures 5-mm tall, 8-mm wide and 10-mm long \cite{design}. This allows the robot to enter through small incisions made beneath the sternum. With this approach, direct access to the heart is provided whereas the doctor would be required to deflate the left lung to access the heart. A second incision can be made in the \textit{pericardium} (the sac enclosing the heart), and the surgeon can place the robot on the surface of the heart by hand \cite{design}. Due to the small-scale of the robot, this offers advantages over current practices by avoiding cardiac stabilization, lung deflation, and access limitations. The robot improves precision and stability and decreases disease risk associated with access \cite{mellon}.
\newline
\indent As with human centered robots, the HeartLander is a robot that can move independent of human interaction. The HeartLander imitates worm-like movement using the two section body described earlier. A vacuum line is supplied through the tether to provide suction as its means of traversing the \textit{epicardial} (outer) surface of the heart. The use of suction to adhere to the surface of the heart is an FDA-approved technique used in medical devices that stabilize the heart \cite{design}. A wire transmission also runs through the tether to off-board motors to control the extension, suction, and retraction of the locomotion capable from the robot \cite{design}. Once place on the surface of the heart, this crawling robot uses these methods to guide itself along without problems.
\newline
\indent A vital characteristic of human centered robots as well as medical robots is the ability to sense the environment. This is achieved through the tracking sensor located in the front body section of the robot. The tracking sensor uses a miniature magnetic tracking sensor called microBIRD from Ascension Technology \cite{design}. Real time positioning is displayed in a 3D GUI from the sensor's position data. From this GUI, the surgeon can either control the robot with a joystick or specify a target location and let the robot navigate itself. Having the robot navigate itself leads to another characteristic of human centered robots.
\newline
\indent To be classified as a human centered robot, the robot must have the ability to think. The HeartLander robot has the capability to navigate autonomously if requested. Surgeons can specify a location on the heart and have the robot crawl towards the area without intervention from the surgeon. During the time it takes the robot to reach to requested area, surgeons can use this to plan the next course of action without needed to concentrate on guiding the robot to the designated spot.
\newline
\indent One last characteristic of the HeartLander is the ability to manipulate tools for different epicardial interventions. To this date, surgeons have percutaneously used the HeartLander to perform dye injections and epicardial pacing lead placement \cite{design}. The HeartLander is equipped with a 2-mm working port where the tools are deployed for epicardial interventions. Dye injections are performed by the surgeon using a remote system to inject patterns into the myocardium. After testing, hearts were excised and the dye injection depth was measured. The average depth was 3-5 mm with no harmful hemodynamic and electrophysiologic events \cite{results}. Pacing lead placement was also tested after a surgeon navigated the robot to the left ventricle of a beating porcine heart and secured the pacing lead to the myocardium. The success of the pacing lead placement was confirmed by fluoroscopy and actual electrical pacing tests with no harmful hemodynamic and electrophysiologic events as well.
\newline
\indent The HeartLander is a significant robot invention that has several advantages over current methods used in cardiac surgery. Providing a method to perform minimally invasive surgery reduces many risks that will preserve the well being of patients who undergo cardiac surgery with a HeartLander robot. With the rapid development of robots, the medical world is changing from open chest cavities to two small incisions just below the sternum. 

% Overview - http://www.cs.cmu.edu/~heartlander/
% Design - http://www.cs.cmu.edu/~./heartlander/design.html#Locomotion
% Results - http://www.cs.cmu.edu/~./heartlander/results.html
% Carnegie Mellon - http://www.ri.cmu.edu/research_project_detail.html?project_id=533&menu_id=261
\end{document}  