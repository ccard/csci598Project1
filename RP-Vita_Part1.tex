\documentclass[11pt,conference]{IEEEtran}
\usepackage[utf8]{inputenc}
\usepackage[english]{babel}

\begin{document}

\section{RP-Vita Part 1 (Johnathan Aspinwall)}

RP-Vita is a revolutionary healthcare robot because it provides doctors with the ability to effectively be in multiple places at once. One of the best features of this device is video-conferencing with multiple specialists. Many of the complications with illness come from miscommunication that could be avoided if the doctors were all present at the same time. This device provides that oppourtunity, saving many lives from being unnecessarily lost. Another video-conferencing option is if a child is rushed to the emergency room from a day camp or school they still need a parent's signature to authorize treatement; RP-Vita can conference with a parent to speed up the process. It also offloads some of the more redundant tasks from the nurses, as well as providing them with reminders when needed. It also provides instant access to patient information, which not only helps prevent errors caused by unavailable information but also helps to maintain more accurate information on the patient by keeping it up to date and immediately correcting any errors.
\newline
\indent The way RP-Vita perceivs and acts is part of what makes it so revolutionary in healthcare. In one setting it can act passively, as the doctors converse and interact with patients, but in another setting it can act autonomously, using facial recognition or RFID to identify and find patients in large hospitals or navigating busy corridors after a nurse directs it to a room at the touch of a screen. The robot uses a number of methods to accomplish everything, including object detection and avoidance, facial recognition; iRobot has not released any technical details as to how exactly they accomplish this. In addition to the navigational sensors, RP-Vita contains numerous medical sensors including devices like a digital stethoscope, so a doctor can remotely listen to the heart and lungs. It also includes an otoscope and an ultrasound.

\end{document}