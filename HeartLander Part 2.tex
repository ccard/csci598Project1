\documentclass[12pt,conference]{IEEEtran}
\usepackage[utf8]{inputenc}
\usepackage[english]{babel}

\begin{document}

\section {HeartLander (Vicente Gonzalez)}

\indent The Heartlander is not intelligent, while it can autonomously navigate to a specified location it cannot learn and comprehend what it is doing. There are no complex algorithms which give the robot intelligence to learn. Although this is the case, the robot is semi-adaptive; it can walk autonomously to a specified location given by the surgeon and begin its therapeutic applications. It does this by using a pure pursuit-tracking algorithm designed to navigate the bot to predetermined surface targets. This limited adaptation is important in the application because it robot to adapt to its environment by using its sensors. Surgeons can also take control of the robot if need be and fully maneuver the bot with a joystick and command it to administer the therapy. Heartlander is currently in its pre-clinical development stage.
\newline
\indent In preclinical development, the robot was still large and hooked up to a tether. The robot has not incurred adverse hemodynamic or electrophysiologic events, but the bot is being developed so that it can one day be about 3mm wide instead of 8mm. It can’t be reused, and is thus disposable. Currently the designers are trying to make the robot tetherless because the tether causes the bot to be rigid and if it were tetherless then it could increase its turning efficiency. The limitations of the robot also include the amount of procedures it can administer. At the present time it can only apply two specific procedures: placing lead placement and Myrocardial Injections. The technology is being further developed so that the Heartlander can administer more than a couple procedures for a patient. 
\end{document}